\documentclass[letterpaper,oneside,titlepage]{article}
\usepackage[utf8]{inputenc}
% ------------------------------------------------

% Selected Packages
% ------------------
% Allows the use of standalone input documents 
\usepackage{standalone}
% Lets me write the dates nicely
\usepackage[super]{nth}
% Basic packages that are good to use
\usepackage[hidelinks]{hyperref}
\usepackage{amsmath}
\usepackage{amsthm}
\usepackage{amssymb}


% This is the main file for a Notebook I will be keeping while working on my thesis
% it will contain notes of what I do on given days, and also contain a running bibliography
% as well as notes on any research and readings I have done

%--------------------------------------------------
\begin{document}

\title{CS Thesis Notebook}
\author{Andrew Hilton}
\maketitle


% ----------------------------------------------------
\section*{Week of October \nth{2}, 2016}

\subsection*{Goals for the Week}
\subsubsection*{Readings}
\begin{itemize}
\item Read John's paper on the Face-encoding Grammar \cite{Rieffel:face-grammar}
\item Read the paper Pete sent me on math and stuff
\item Look at citations of Toussaint's paper
\cite{toussaint2003demonstrating}
to see if there has been any work adapting his style of grammar to a similar problem space
\item add citations to those reading things, just so everything is nice
\end{itemize}
\subsubsection*{Organization}
\begin{itemize}
\item Figure out a weekly time to meet with John
\item Come up with a note taking template to include in the notebook
\item Add new sources to centralbib
\item Retroactively digitize the contents of my physical notebook
\end{itemize}

\subsection*{Monday, October \nth{3}}
\subsubsection*{General Notes}
\begin{itemize}
\item Finally started this notebook.  Going to have to figure out how to do stuff when I don't have internet access.  Probably going to look into implementing some type of include or input command, and just keep weekly documents and compile them into this central document.
\item Finally got internet access on my laptop up campus.
\end{itemize}

\subsection*{Tuesday, October \nth{4}}
\subsubsection*{Objectives for Today}
Today I would like to read and take notes on the paper outlining John's tetrahedral face-encoding grammar.  This means that I need to \textbf{A)} Actually add that paper to the bibliography, and \textbf{B)} work on making my note taking template, and append that to the notebook.

\subsection*{Wednesday, October \nth{5}}
Did a quick read-through of John's Face-Encoding Grammar paper \cite{Rieffel:face-grammar}.  Need to do a more in-depth note-taking read through, but the main focus of the paper seemed to be concerned with simulating the soft robots, not the uses of the grammar itself.  I will have to talk with John in person a little bit more about the grammar

\section*{Thursday, October \nth{6}}
\subsubsection*{Things I Have Done Today}
\begin{itemize}
\item Signed up for GitHub
\item Added John's Face-Grammar paper to centralbib
\end{itemize}
\subsubsection*{Things I Should Do Today}
I should setup a weekly meeting time with John, and try to talk with him about the next steps in my project.  Now that I have a GitHub account, I might move this notebook and the work to a repository there so that I can do more stuff from command line, and not have to deal with Overleaf.  If I have time today I am going to try and find some more papers talking about systems for producing Generative Encodings, and add some sections to the notebook about work I did before starting the tex notebook.  Tomorrow I would like to do some more note-taking on papers.
\subsubsection*{Lab Meeting}
Frankie talked about noise in simulation.  According to John, Jakobi has a good paper for anyone doing stuff on Genetic Algorithms (Frankie is sending that to me over Slack)

% ----------------------------------------------------------

\bibliography{centralbib}
\bibliographystyle{plain}

\end{document}