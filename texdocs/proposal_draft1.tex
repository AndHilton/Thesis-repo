\documentclass[condensed]{union-cs-thesis}

% made a fresh document to write the draft

\usepackage{graphicx}
\usepackage[hidelinks]{hyperref}
\usepackage{cite}

% these packages are being used to allow for margin notes for editing
\usepackage{marginnote}
\usepackage[top=1.5cm, bottom=1.5cm, outer=5cm, inner=2cm,
  heightrounded, marginparwidth=4cm, marginparsep=.5cm]{geometry}
% new command that makes leaving notes in the margin a little more friendly
\newcommand{\editnote}[2]{\marginnote{\footnotesize #1}[#2]}
  

%\graphicspath{./images/}https://preview.overleaf.com/public/jfrqpjrmtgyz/images/6b502d7641596bc8fc5625d0a458ebb8588298ac.jpeg

%------------------------------------------------------------
\begin{document}

\title{Generative Models of Biological Ontogeny:\\\Large Evolving ``Skate Guts''}
\author{Andrew Hilton}
\date{\today}

\maketitle

\editnote{not really sure if I should be capitalizing Generative Encoding and Genetic Algorithm
  throughout}{0cm}
%------------------------------------------------------------
\begin{abstract}
\makeabstract

\textbf{TODO: Rewrite the Abstract. This is a \emph{really} rough draft of the abstract}

\vspace{.5cm}
Genetic Algorithms ({\em GAs}) have been proven to be incredibly powerful tools in many different areas,
from creating novel designs to automatically generating solutions to complex computational problems.
Biological studies especially have found a lot of applications for genetic algorithms.  One possible
use is in evolving models of developing biological systems.  I hope to be able to use Generative
Encodings in order to accurately model the development of skate intestines.

\end{abstract}

%------------------------------------------------------------
\section{Introduction}

\textbf{TODO: Rewrite Intro.  This is the intro I gave to Aaron a couple of weeks ago.  I need to sit
  down and sort through his comments, and also revisit this myself but for now I think I want to get
  some comments on the rest of the proposal.  Some of this information is actually inaccurate, so I
  really just putting this in here to start to fill up the space a little more}

\vspace{.5cm}
As the power of our medical technology has increased, so to has our ability to understand the workings
of Biological Systems. Many things that previously would have been impossible to model, or even observe, 
have become simple tasks with modern advancements in medical imaging. One such area is the evolution
of animal digestive systems over the course of their early development. An animal with some particularly
interesting insides is that of the Skate, a type of kite-like fish similar to a Manta-Ray
\footnote{to add: skate picture}.
\par
Skates are particularly intriguing based on their age (clocking in at nearly \emph{4.5 billion years})
and their diet. Like Humans, Skates' diets are very high in protein, which is a notoriously difficult
group of macro-molecules to digest, requiring a lot of surface area to absorb the necessary nutrients.
Humans evolved to solve this problem by having long digestive tracts, that coil around themselves for
more than twice ouraverage height.  Skates however, did not have this luxury during their evolution.
Being invertabrates, with no real bones in their body, their adbominal cavity (the space inside them)
is not large enough to supporta long, coiled digestive system. Instead, they developed a small
intestinal organ that spirals around withinitself in order to provide the necessary surface area
\footnote{to add: picture of skate guts}.

\par
\emph{It might be worth introducing the concept of genetic algorithms in general either here or in
  the related  works section.  That would be something to talk about with Aaron}

\par
Using CAT scans, it is possible to create 3D models of these systems, even during the embryonic phases
of development.Seemingly in order to model the full development of the digestive systems throughout this
time, it wouldbe necessary to take near constant scans of the embryo throughout.  However, using the
concepts of Genetic Algorithms it may be possible to forgo much of that process, by ``evolving'' models
as opposed to scanning them.  A lot of work has been dfone in using Generative Representations to create
new models for practical applications\cite{rieffel2009automated}, as well as more novel creations
\cite{hornby2004functional}, even to model simplerartificial biological structures
\cite{toussaint2003demonstrating}.  By applying these techniques to a model of ``skate guts'' and then
evolving its structure, it may be possible to simulate and model the full developmentof the system,
using only a small number of reference models.

%------------------------------------------------------------
\section{Related Works}

\par
While the use of Genetic Algorithms is nothing new within the field of Computer Science, and their
applications are continually being investigated and expanded upon, this particular type of application
appears to be largely unexplored.  Even though GAs have long been used in various fields, and Biology
is far from being an exception, it does not appear as though much research has been done into
creating evolutionary representations of developing biological systems.  However much work has been
done in exploring the potential uses of Generative Encodings.  Various investigations in the field of
robotics have shown that Generative Encodings can be used to develop novel designs for soft-robots
\cite{rieffel2009automated},
as well as generating designs for antennae to be used in various satellite
\cite{lohn2005evolved}.
As was shown in Greg Hornby's work in developing a system for the creation of novel table designs, the
core concept behind Generative Encodings allows for the system to evolve encodings that are much more
complex, and as a general rule much more viable in terms of matching the physical constraints of the
objects themselves
\cite{hornby2004functional,hornby2001advantages}.

\par
\editnote{Not really sure if I want to include this paragraph on Bongard as I don't really know
  how related it is to my topic overall.  I do think it is interesting to see that the idea of
  ontogeny already was being used in genetic representations, but this is mostly related by conflation
  of meaning.  Probably will take it out next draft, just wanted to get some of these thoughts on paper}
{-3cm}
Some of the work done by Josh Bongard deals with an interesting subset of evolutionary design,
namely the area of ``artificial ontogeny,'' i.e. using the concepts of generative encoding to find
solutions to various tasks by evolving artificial pseudo-organisms
\cite{bongard2002evolving}.
Despite it's surface-level relation to my topic area, most of Bongard's work has been in the field
of modular robotics, which uses small identical robotic ``cells'' to create functional multi-celled
robots that are capable of self-modifying in order to adapt to different situations.  While this is
not directly related to the proposed project, it is interesting to see that the idea of ontogeny is
already being used in the field of evolutionary representations.

\par
Despite this specific type of problem not being widely investigated, some of the work done by Marc
Toussaint has some overlap with the type of work planned for this project.
\editnote{Don't really like the start to this paragraph, but will revise later}{0cm}
Toussaint's study hasshown that evolving a grammar-based encoding of artificial plants produces more
effective solutionswithin fewer generations on average than a direct encoding
\cite{toussaint2003demonstrating}.\footnote{in this case the ``problem'' that solutions were trying to
  solve was how to maximize the plant's exposure to the sun}
The underlying thought behind the work done by Toussaint was the idea that a single phenotype
\footnote{the physical manifestation of genetic encoding}
could be represented by more than one genotype.
\footnote{a string of genetic information, a specific type of gene}
Toussaint was able to show that using a Genetic Encoding of an object allowed for multiple solutions
of similar shape be evolved, while having different underlying genetic represenations.  His work also
showed that using Generative Representations increased the overall effectiveness of the evolved
solutions, generating more fit solutions within fewer generations.  This lends itself to the strength
of Generative Encodings in regards to the proposed project, as the desired phenotype is already more
or less known, so the question lies more in what are the genotypes that lead to this trait.  Toussaint's
work also deals with grammar-based representations, which are of interest as we look to model the full
development of the skate intestine.  However the representations of plants he uses may be too simple
to be fully translatable to the more complex shapes of the skate organs.

%------------------------------------------------------------
\section{Planned Work}

%---------------------------------------------
\subsection{Approach}
\editnote{I don't really know how much more I can say about the approach without specifics about the
  actual system/approach I will be using to make the models}{0cm}

At this point I have a fairly strong idea of the overall goal of the project, but the specifics of how
to approach actually creating the evolvable models will take some dedicated research.  Due to the nature
of the project, a lot of the work to be done is in deciding how to create the representations of the
models.  There are many methods that are used in the creation of Generative Encodings.  Some of them
range from more mathematical representations that rely on graphs where nodes and the vertices between
them represent links between different parts of the structure
\cite{rieffel2009automated}.
This type of representation is more applicable to the problem space of designing tensegrities.
One of the more classic examples is the  grammar-based, turtle representation.  In this method a
grammar of actions is defined, and sentences within this grammar dictate the path that will be taken by
a ``turtle,'' which places 3D building blocks as it travels
\cite{hornby2004functional}.
There are other forms of grammar representations that don't strictly rely on the turtle method
\cite{toussaint2003demonstrating},
however many of the forms that are created through these grammar defined representations are either
too simple or too rigid to be of use for our needs, though there is still a lot of investigation to be
done before this can be determined.  Another type of representation that shows some potential are
\emph{Compositional Pattern-Producting Networks} (or CPPN)
\cite{clune2010investigating}.
\footnote{These are a form of Neural Network that uses linear combinations of Sigmoid and Gaussian
  functions torepresent complex patterns.  This type of network is used in the HyperNEAT program.}

\par
The first real step for my project will be in investigating through further research and my own
experimentation which type of tools and representation will be best suited for my project.  Once
this has been decided the next step would be to begin running tests, to see what kind of initial shapes
get evolved.  I am interested in investigating whether or not the evolutionary constraint that is
believed to have led to the particular shape that is observed in skates
\footnote{they need a lot of internal surface area, but have confined possible volume}
can actually be used to evolve similar shapes in the context of a Generative Encoding.  Therefore in
the initial runs, the evolvedmodels will be fitness-tested by a fairly simple geometric relation between
the volume occupied by the shape, and the internal surface area contained within the volume.  The hope
is that this will lead to twisted, spiraling shapes, like those seen in skate intestines.  Following
these initial runs (which will most likely require some kind of fine-tuning in the initial stages), I
will continue to refine the algorithm I am using by altering the fitness function used to evaluate the
models, but also making alterations to the representation itself based on the results of previous runs.
I will continue this process of doing runs and refining until I am satisfied that I am ready to start
doing analysis on my models in comparison to the models produced by the Biology department.

%---------------------------------------------
\subsection{Evaluation}
\editnote{I can't really tell to what extent I am ``experimenting'' for this project, as in most
  of the studies I have read dealing with this kind of topic have done some kind of comparison
  between the effectiveness of an evolved direct encoding vs. a Generative Encoding.  I guess in my
  situation the comparison to judge the effectiveness is just going to be based on the geometric
  analysis because there will already be a ``control'' of sorts in the form of the scan-based models}
  {0cm}

  Given that, unlike a lot of the notable work done with Generative Encodings, there is a real,
  observable, biological counterpart
  \footnote{with their own 3D models}
  to the models I will be developing, the evaluation of my work will be fairly straightforward to
  a degree.  When Professor Theodosiou and her students have finished performing their scans and
  generating the 3D models of the physical skate intestines, I will be able to directly compare my
  evolved models to theirs.  The challenge to this part of the process will be in determining the best
  course of action to perform this comparative analysis.  There have been some papers within the
  field of Mathematics that were written with the intent of developing a methodology of performing
  geometric comparative analysis of 3D objects in order to compare their ``likeness''
  \cite{shum1996:3d}.
  It is not clear to me at this time as whether or not the formulae discussed in the aforementioned
  paper will be sufficient to provide conclusive results to evaluate the 3D shapes in my project,
  or if it is even feasible to use such formulae to perform this kind of comparison on shapes of this
  complexity.  As such, at some point in the future I will be consulting faculty within the Math
  department here at Union to gather their insight on the best way of approaching this kind of problem.

%------------------------------------------------------------
\section{Timeline}

Due to the collaborative aspect of my project, I am somewhat bounded by the work that Professor
Theodosiou and her students are able to do.  Fortunately much of the work they will be doing is
going to happen over the Summer, and a lot of the early work for my project can be largely
independent from the models.

\begin{description}
\item[Fall] During the Fall I will be continuing background research into the topic.  A lot of the
  research at this point during my project will be in investigating different software and
  techniques for creating and evolving Generative Encodings.  As stated previously I have encountered
  a couple of potential systems used by others for somewhat related investigations, so during the
  Fall I would be doing some experimentation to see which of these could best be adapted for my
  particular project.
  \par
  Once this has been decided I can start running preliminary trials, using simple fitness functions
  in order to start refining the system.  I will be running multiple trials and making changes to the
  system.  If by the end of the term I feel confident that I have made significant progress, I will
  begin consulting some of the Math department faculty to get their insight on how to approach the
  geometric analysis of the models, in the hope that they will be able to point me in the direction
  of good resources for those types of problems.
\item[Winter] Forgoing serious setbacks in their research, Professor Theodosiou and her students
  will hopefully have completed all of the scans that they need and created the 3D models of the
  various phases of development that they are hoping to model.  Thus, depending on my progress during
  the Fall and how far I was able to get in finding a geometric analysis formula, I will begin
  analyzing all of my runs by comparing the models evolved by my system, and the actual models created
  from the CT-scans.  Depending on time constraints and the overall progress I have made at this point
  I will continue refining my system using the geometric analysis to inform my decisions.
\end{description}


\bibliographystyle{plain}
\bibliography{centralbib}

\end{document}
