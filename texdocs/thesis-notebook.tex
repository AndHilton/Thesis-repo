\documentclass[letterpaper,oneside,titlepage]{article}
\usepackage[utf8]{inputenc}
% ------------------------------------------------

% Selected Packages
% ------------------
% Allows the use of standalone input documents 
%\usepackage{standalone}
% Lets me write the dates nicely
\usepackage[super]{nth}
% Basic packages that are good to use
\usepackage[hidelinks]{hyperref}
\usepackage{amsmath}
\usepackage{amsthm}
\usepackage{amssymb}


% This is the main file for a Notebook I will be keeping while working on my thesis
% it will contain notes of what I do on given days, and also contain a running bibliography
% as well as notes on any research and readings I have done

%--------------------------------------------------
\begin{document}

\title{CS Thesis Notebook}
\author{Andrew Hilton}
\maketitle


% ----------------------------------------------------
\section*{Week of October \nth{2}, 2016}

\subsection*{Goals for the Week}
\subsubsection*{Readings}
\begin{itemize}
\item Read John's paper on the Face-encoding Grammar \cite{Rieffel:face-grammar}
\item Read the paper Pete sent me on math and stuff
\item Look at citations of Toussaint's paper
  \cite{toussaint2003demonstrating}
  to see if there has been any work adapting his style of grammar to a similar problem space
\item add citations to those reading things, just so everything is nice
\end{itemize}
\subsubsection*{Organization}
\begin{itemize}
\item Figure out a weekly time to meet with John
\item Come up with a note taking template to include in the notebook
\item Add new sources to centralbib
\item Retroactively digitize the contents of my physical notebook
\end{itemize}

\subsection*{Monday, October \nth{3}}
\subsubsection*{General Notes}
\begin{itemize}
\item Finally started this notebook.  Going to have to figure out how to do stuff when I don't have internet access.  Probably going to look into implementing some type of include or input command, and just keep weekly documents and compile them into this central document.
\item Finally got internet access on my laptop up campus.
\end{itemize}

\subsection*{Tuesday, October \nth{4}}
\subsubsection*{Objectives for Today}
Today I would like to read and take notes on the paper outlining John's tetrahedral face-encoding grammar.  This means that I need to \textbf{A)} Actually add that paper to the bibliography, and \textbf{B)} work on making my note taking template, and append that to the notebook.

\subsection*{Wednesday, October \nth{5}}
Did a quick read-through of John's Face-Encoding Grammar paper \cite{Rieffel:face-grammar}.  Need to do a more in-depth note-taking read through, but the main focus of the paper seemed to be concerned with simulating the soft robots, not the uses of the grammar itself.  I will have to talk with John in person a little bit more about the grammar

\subsection*{Thursday, October \nth{6}}
\subsubsection*{Things I Have Done Today}
\begin{itemize}
\item Signed up for GitHub
\item Added John's Face-Grammar paper to centralbib
\item Moved all of my documents into Git Repository
\end{itemize}
\subsubsection*{Things I Should Do Today}
I should setup a weekly meeting time with John, and try to talk with him about the next steps in my project.  Now that I have a GitHub account, I might move this notebook and the work to a repository there so that I can do more stuff from command line, and not have to deal with Overleaf.  If I have time today I am going to try and find some more papers talking about systems for producing Generative Encodings, and add some sections to the notebook about work I did before starting the tex notebook.  Tomorrow I would like to do some more note-taking on papers.
\subsubsection*{Lab Meeting}
Frankie talked about noise in simulation.  According to John, Jakobi has a good paper for anyone doing stuff on Genetic Algorithms (Frankie is sending that to me over Slack)
\subsubsection*{Talk with John}
After lab meeting today I forced John to talk with me about my project.	 He is suggesting that I use the tetrahedral face-grammar \cite{Rieffel:face-grammar} going forward, but that I should port the existing code into python for ease of use.  He is going to send me the C code that he has somewhere, afterwards I am going to rewrite the system in Python.  The basic idea is that the subdivide operation is going to be removed from the grammar for ease of use (it made certain parts to complicated).  After that the basic idea of the system is that there is a queue of the the faces (including labels) and every iteration you dequeue the first face, apply the appropriate rewrite rule and queue that into the next iteration's queue.  The system itself really only needs to concern itself with each face, and the points it is associated with.  This type of system would work well with STL format files, but Blender has some interfacing with Python so it would be good to look at both of those for the visual representation of the grammar.  Next steps are to work on those pieces, and start to come up with questions myself.

%----------------------------------------------------------

\section*{Week of October \nth{23}, 2016}
I have a lot to fill in for stuff but I will get to that eventually.

\subsection*{Tuesday, October \nth{25}}
Starting to finally get myself organized on doing the actual setup of the grammar system.
\subsubsection*{Math-stuff}
\begin{itemize}
\item figure out the vector from one corner of a face to the center of the triangle
\item figure out how to find the height of the tetrahedron (gives me the scalar to multiply cross product by)
\end{itemize}
\subsubsection*{What I got}
The overview of my solution of how to calculate the new point after a grow operation has been used.  Pick an edge of the face that is being operated on, and zero it by subtracting the vector from the origin to the corner ($\overrightarrow{p}$).  Next take the cross product of the two vectors that define the edges of the corner. Then multiply that vector by the scalar representing the height of the tetrahedron ($h$) \footnote{make sure that the orthogonal vector is going in the correct direction from the plane}.  Then add the vector to the center of the face ($\overrightarrow{c}$).  This should produce the extended point from the face, after it has been zeroed.  Then re-add the positional vector ($\overrightarrow{p}$) to get the extended vector centered on the original face.
\par
Turns out that this approach of taking the cross product, scaling and adding the vector to the center is \textbf{not the right solution}.  The process of finding the vector to the center is exactly the same process as solving for the system of equations where the dot products are $\frac{1}{2}$.  I will explain more below.

\subsection*{Wednesday, October\nth{26}}
\subsubsection*{I Did It?: Finally solving the Problem of the Grow Operation}
So it turns out that the worst possible thing that could have happened happened, John was right...  After consulting many math professors \footnote{Prof. Jeff Jauregui, Prof. Brenda Johnson} and several STEM faculty and students \footnote{Shelia Kang, Aaron Cass, Pete Johnorson}, and getting many different potential solutions, through many variations, I was able to arrive at the solution proposed by John (although very indirectly and as vaguely as possible).  I will now attempt to be as exact in defining the solution to the math behind the Grow operation.  I will try and talk through the process it took to arrive at this solution in greater detail at a later date.
\subsubsection*{The Thorn in My Side}
Performing the Grow operation on Face $F_0$, defined by the points $V_1, V_2, V_3$, a new point $V_4$ is produced, orthogonal to the plane described by $F_0$.  This produces three new faces $F_1, F_2, F_3$, defined by points $V_1 - V_4$.  $F_0$ is functionally removed from the system.  In order to implement this operation, a way must be found to determine the location of $V_4$, given only the information intrinsic to $F_0$.
\par
The approach to this problem relies on finding the vector orthogonal to the plane by taking the cross product of two vectors that describe the originating face.  Then the vector must be scaled by the height of the tetrahedron that will be produced by the new faces, and repositioned to the center of the originating face, producing a point that is perpendicular to the center of the base, and a unit distance away from the three describing points.  An additional positional shift will have to be performed before and after these calculations in order to compensate for the coordinates of the originating face.
\par
Below is a more in depth description of the process.\footnote{\emph{Disclaimer: This relies on the assumption that the center of an equilateral triangle can be found by taking the average of the 3 points, which I have not verified}}
\par
Given that $F_0$ is offset from its closest point $V$ to the origin by the vector $\vec{p}$, subtract $\vec{V} - \vec{p}$ to center the point on the origin.  Then define two vectors $\vec{u}, \vec{v}$ from the origin to the other points on the face.  The cross product of these vectors $\vec{u} \times \vec{v}$ produces the vector orthogonal to the plane described by the face.  This is then scaled to be the height of the tetrahedron $h = \sqrt{\frac{2}{3}}$.  This is then added to the vector $\vec{c}$, which is defined as the point at the center of the triangular face \footnote{found by taking the average of the 3 points defining the triangle}.  This should then produce the vector $\vec{w}$ with length $h$, perpendicular to the face, at the point above the center.  The original positional vector $\vec{p}$ is then readded to this value in order to reposition back on the originating face.  Using this newly recentered value for $\vec{w}$ a new vertex can be created, along with three new faces.
\subsubsection*{Moving Forward}
Having finished determining the relationships in order to implement the Grow operation I can now start moving forward with the implemention of the system itself.  I will need to find some python packages that will allow me to calculate the cross product (otherwise I must implement that myself) and find some classes for the points and vectors with the necessary operations already implemented.  I can then begin implementing my own additions to the system (i.e. creating the face class, and the actual computational aspect of the system).  I will then work on integrating these structures into an existing modeling software and then begin on implementing the evolutionary component of the project.  I would like to be done with the system implementation and some rudimentary integration with modeling software by the end of the week, so that I can sooner rather than later begin collecting some results.
\subsubsection*{Beginnning to Implement}
I started to implement some of the classes.  I am most likely going to be relying heavily on NumPy, as it is readily available, and it seems to cover most of my needs.  I am going to be using NumPy arrays as the underlying representation of the Vertices, but I wanted to wrap it in something so that I could make changes and tweak the interface.  Each face has an list of 3 vertices that it, and its label.  Need to figure out some of the exact details of the implementation (such as how the vertices are stored).  One of the decisions I have to make for the system at large is that I need to decide whether or not to create a new class for the grammar rules themselves, or to keep that part of the system simplified as a library.  I will just use the default list as a queue because the functionality is already built in to those as a data structure.

\subsection*{Thursday, October \nth{27}}

\subsubsection*{Bio-Seminar}
John told me that the Bio seminar for this week might be related to my thesis in some way.  It is one of Professor Theodosiou's colleagues, who is doing work on developmental biology, specifically how organs develop.  \emph{Molecular control of physical forces during morphogenesis of the vertebrate gut}\footnote{apparently gut is considered a biology term} by Nandan Nerurkar.
\par
Developmental Biology is the study of how organisms go from embryonic cells to fully formed organisms (ontogeny).  We care because when things go wrong we get birth defects.  There is the idea that any change in shape requires a physical force.  Originally we thought that human development was just getting bigge (sperm cells were thought to contain just small people who got bigger).  However we have since learned that there is actually a transformation occurring.  D'arcy Thompson: On Growth and Form (1917), D'arcy outlined the importance of mathematics and physics in the study of biology.
\par
Studies have shown that there is feedback between physical forces and genes.  Genes are able to create forces in order to change the morphology of physical tissues, but physical forces (such as magnetic) can actually affect the way that genes get expressed.  The main question of Nerurkar's research is to figure out how developmental signals modulate physical forces to shape developing embryo, specifically by looking at the small intestine, because it is important, and interestingly shaped.  The intestines look like they are jumbled together randomly, but they are actually very specifically arranged.  This is shown by the number of loops in the small intestine being conserved throughout members of a species, independent of size.  Because the shape is important, when the process of looping goes wrong, it can lead to serious birth defects in infants.
\par
\textbf{The Morphogensis of the Gut:}  The digestive system starts off as a single tube that encompasses all of the necessary functions, although not well.  As the gut continues to develop different regions of the tube begin to separate into different organs, and begin to take the shape necessary for their specific function.  They were able to observe that there is a physical interaction between two types of tissue in order to create the looping of the intestine.  There is a membrane on the side of the tube, that grows at a slightly slower rate than the tube itself, which causes the looping.
\par
I kind of stopped paying attention for a bit, but it seems like they were using gene inhibition and different viruses and bacteria to try and create physical changes in the system in order to affect the shape that the organs develop, or were able to observe the developmental signals.  They then used the information on the developmental signals to study the physical forces at play in development, and how the two things interact.

\subsubsection*{Implementation Work}
I think the next step is to finish implementing the face and vertex classes, test them, and begin working on the actual grammar production system.  I still need to answer a couple of design questions, but I think that I can be pretty successful with a fairly basic system.  I am probably going to just implement the rules and operations as a dictionary mapping labels onto a tuple containing the function and the list of arguments to the operation.  A question I will have to answer once I get to the evolutionary component of the system (probably worth talking about sooner rather than later) is how the number of arguments to a given operation will be determined automatically.  \emph{Is it possible to ask a function (as a first class object) how many arguments it takes?}  I then need to work on the modeling interface, which will require looking into the modeling software I have available to me.  I know that there is a module that allows you to interact with STL fairly easily.
\\
\textbf{Questions to Answer:}
\\
\begin{itemize}
\item Where will the calculations for vertex location be located?  It is mostly used for the grow operation, but it will be needed for the setup.
\item How will the initial state of the system be setup?  Will it be populated by multiple tetrahedra, or will it just start with a single tetrahedron with a random assortment of labels and proceed from there?
\item Should the initial setup of a tetrahedron rely on the grow operation?  That would basically bring the amount of information needed to initially setup the system would be initializing a single face, and picking a direction for the cross product.
\item How will the positional vector $\vec{p}$ be determined?  Initially I figured it should just be the one closest to the origin, but would that make the calculation more difficult if that cross product goes in the ``wrong direction''?  I think the best way of doing this would be to find the normalized cross product, and multiply it by $h$ and $-h$, in order to find both directions.  Intuitively it seems like the vertex extension calculations in this case would produce 2 vertices, one that already existed (opposite the face being operated on) and one that does not, which is the new one that gets produced by the operation.  However there may be some problems if somehow two faces end up being parallell to each other.  I don't actually know if it is possible to have that kind of collision in a tetrahedral mesh system, that might be why people like them.
\end{itemize}


% ----------------------------------------------------------

\bibliography{centralbib}
\bibliographystyle{plain}

\end{document}
